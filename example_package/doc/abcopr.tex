\documentclass[opr]{talentTex}

\title{Przykładowy tytuł zadania}
\id{abc}

%%%%%%%%%%%%%%%%%%
% Przydatne komendy:
% \pagebreak % komenda zaczynająca nową stronę pdfa
% ~ % słowa oddzielone '~' zamiast ' ' w~taki sposób będą zawsze koło siebie
% $text_mat$ % pozwala na pisanie wyrażeń matematycznych
% Dokumentacja:
%   https://www.overleaf.com/learn

%%%%%%%%%%%%%%%%%%
% Komendy talentowe
% \start{}  % Rozpoczyna treść, musi być na samym początku treści opeacowania.
% \finish{} % Kończy treść, musi być na samym końcu treści opeacowania.
% \tSection{text} % Nagłówek w stylu talentu.
% \tCustomSection{text}{xpt} % Nagłówek w stylu talentu, z możliwością ustawienia odstępu 'x' od poprzedniego akapitu.
% \tSmallSection{text} % Mały nagłówek w stylu talentu.
% \tc{text} % Styl używany do oznaczania zmiennych.

\start{}

\tSection{Pochodzienie zadanie}


\tSection{Poziom zadania, trudnoości}


\tSection{Podzadania}


\tSection{Istniejące rozwiązania}


\tSection{Jak wygenerowano testy}


\tSection{Ustawienie limitów}


\tSection{Inne}


\finish{}
