\documentclass[12pt]{article}
\usepackage{geometry} 
\geometry{a4paper} 
\usepackage{amssymb}
\usepackage{polski}

\title{Opracowanie}
\date{} % pozostaw puste

%%%%%%%%%%%%%%%%%%
% Przydatne komendy:
% \pagebreak % komenda zaczynająca nową stronę pdfa
% ~ % słowa oddzielone '~' zamiast ' ' w~taki sposób będą zawsze koło siebie
% $text_mat$ % pozwala na pisanie wyrażeń matematycznych
% Dokumentacja:
%   https://www.overleaf.com/learn

\begin{document}
\maketitle

\section{Poziom zadania, statystyki}


\section{Ustawienie limitów}


\section{Istniejące rozwiązania}


\section{Jak wygenerowano testy}


\section{Inne}


\end{document}