\documentclass[12pt]{article}
\usepackage{geometry} 
\geometry{a4paper} 
\usepackage{amssymb}
\usepackage{polski}

\title{Opis rozwiązania}
\date{} % pozostaw puste

%%%%%%%%%%%%%%%%%%
% Przydatne komendy:
% \pagebreak % komenda zaczynająca nową stronę pdfa
% ~ % słowa oddzielone '~' zamiast ' ' w~taki sposób będą zawsze koło siebie
% $text_mat$ % pozwala na pisanie wyrażeń matematycznych
% Dokumentacja:
%   https://www.overleaf.com/learn

\begin{document}
\maketitle

\section{Sekcja}
    Uzupełnij ten plik według własnych upodobań stylistycznych.

    Jedyna ważna rzecz, to aby każdy po przeczytaniu umiał rozwiązać to zadanie.


\end{document}